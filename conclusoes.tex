\chapter{Conclusões}

Levando em consideração os resultados obtidos, pode-se considerar que o
objetivo geral e os objetivos específicos identificados no início desta
monografia foram atingidos. O método proposto apresentou, para o teste
executado, bons resultados, conseguiu classificar as imagens de forma
satisfatória gerando automaticamente os grupos conforme se pretendia, sem,
contudo, exigir um tempo de execução que tornasse impraticável seu uso,
pelo contrário, a análise assintótica do método demonstrou que seu
comportamento é, para o núcleo do processo, linear.

Entretanto, existem pontos onde melhorias são imprescindíveis. Mesmo que o
método não exija a definição explícita do número de conjuntos, a rede ainda
necessita da definição manual dos seus parâmetros, isto é, das dimensões da
malha de saída, da tacha de aprendizado, da largura efetiva da vizinhança
e da constante de tempo. Esses parâmetros também
podem, em certa medida, ser inferidos dinamicamente com base em uma análise
prévia dos dados de entrada, existem sofisticados formalismos destinados a
este propósito. A inclusão deste tipo de técnica deixaria todo o método muito
mais robusto.

Outo ponto crucial que pode ser melhorado são os descritores utilizados para
classificar as imagens. Os momentos de Hu são ótimos para trabalhar
exclusivamente com características morfológicas presentes nas imagens, contudo,
os demais aspectos como cores e texturas são totalmente negligenciados. Uma
combinação dos momentos de Hu com algum tipo de análise de modelos de misturas
gaussianas poderia gerar classes
mais concisas, e permitir o agrupamento de imagens mais complexas.
