\begin{abstract}
When pictures began to be computationally processed and stored emerged
techniques to group them, the same way as has occurred with other types of
digital data, with the aim of discovering interrelationships between elements
through the group to which they were classified. This paper presents an image
clustering where the groups are defined according to shapes in the images and
does not require the explicit definition of the number of classes to be
identified. This technique is divided into four steps, extraction of
descriptors, standardization, Kohonen network training and mapping the images.
In the extraction of descriptors, each image is binarized using Otsu's method
to obtain the overall threshold then is calculated for each binarized image
the Hu invariant moments, a set of seven descriptors that are independent of
rotation, translation or scaling. In step normalization the data are adjusted
so that values not significant and values very significant not have equivalent
weights at the time of classification. In the training phase of Kohonen neural
network is used to provide a self-organizing map where images are classified.
In the last stage, the mapping of images is analyzing unified distance matrix
generated by the Kohonen network connections, using the watershed transform.
This technique showed 78\% accuracy in classification of images for the test
run, and have run the process in a satisfactory time.

\textbf{Keyword}: Image clustering, Kohonen neural network, Hu invariant
moments, watershed transform
\end{abstract}
