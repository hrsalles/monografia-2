\begin{resumo}
Quando imagens passaram a ser processadas e armazenadas computacionalmente,
surgiram técnicas destinadas a agrupá-las, do mesmo modo como já ocorria com
outros tipos de dados digitais, com objetivo de descobrir inter-relações entre
os elementos através do grupo para qual eram classificados. Este trabalho
apresenta uma técnica de classificação de imagens onde os grupos são definidos
de acordo com formas presentes nas imagens e que não necessita da definição
explícita da quantidade de classes a serem identificadas. Esta técnica se
divide em quatro etapas, extração dos descritores, normalização, treinamento
da rede de Kohonen e mapeamento das imagens. Na extração dos descritores cada
imagem é binarizada utilizando o método de Otsu para obtenção do limiar
global, em seguida é calculado para cada imagem binarizada os momentos
invariantes de Hu, um conjunto de sete descritores que independem de rotação,
translação ou escala. Na etapa da normalização os dados são ajustados para que
valores pouco significativos e muito significativos não tenham pesos
equivalentes no momento da classificação. Na etapa de treinamento uma rede
neural de Kohonen é utilizada para fornecer um mapa auto organizável onde as
imagens serão classificadas. Na ultima etapa, o mapeamento das imagens ocorre
analisando a matriz de distâncias unificadas gerada pelas conexões da rede de
Kohonen, com auxílio da transformada de \textit{watershed}. Esta técnica
apresentou, para o teste executado, 78\% de precisão na classificação das
imagens, além de ter executado todo o processo num tempo satisfatório.

\textbf{Palavras chave}: \textit{Clustering} de imagens, redes neurais de
Kohonen, momentos invariantes de Hu, transformada de \textit{watershed}
\end{resumo}
